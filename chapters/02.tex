\chapter{Formal Syntax and Operators}
This chapter is not really meant to be read from top to bottom, only to be used as a reference when reading later chapters. If the way anything is worded is confusing, this chapter is meant to act as a guide.

\section{Relational Operators}\label{sec:relational-operators}
\begin{description}
	\item[$\rightarrow$/$\Longrightarrow$]
		"Produces", "implies", or more generally "when conditions on the left are satisfied, the direct logical outcome is on the right". Used in boolean arithmetic, chemistry, physics.
	\item[$\leftarrow$/$\Longleftarrow$]
		"Produced by", "implied by", or more generally "when conditions on the right are satisfied, the direct logical outcome is on the left". Used very rarely in boolean arithmetic, more commonly: chemistry, physics."
	\item[$\iff$]
		"Produces and is produced by", "implies and is implied by", often "if and only if", or more generally "conditions on the right occur as a direct result of conditions on the left, and not as a direct result of anything else". Used in boolean arithmetic and chemistry.
	\item[$\ne$]
		Not equal; the opposite of $=$. Used everywhere $=$ is used.
	\item[$\approx$]
		Approximately equal. Used wherever $=$ is used.
	\item[$\propto$]
		"Proportional to". Used most commonly in pure mathematics, but can be seen in any discipline that uses $=$.
	\item[$\sim$/$\simeq$]
		"Similar to", "same shape" or "equal in proportion (but not magnitude)". Used primarily in basic geometry.
	\item[$\cong$]
		"Congruent to", "same shape and size", or "equal in both proportion and magnitude". Used primarily in basic geometry.
\end{description}

\section{Boolean Logic}
A basic datatype not covered in Chapter 1, but is also extremely useful is what's known as a 'boolean,' which is represented in Python as the type \code{bool}. This is much simpler to define than a set of numbers, because rather than infinite possible values, there are only two: true or false. This is useful in cases when operations are to be preformed conditionally. For example, the famous Dirac Delta function models the density of a point-mass. A definition for this function may be found in Equation \ref{func:dirac}/Listing \ref{code:dirac}.

\begin{equation}\label{func:dirac}
\delta(x)=
\begin{cases}
0,&x\ne 0\\\\
\infty,&x=0
\end{cases}
\end{equation}

\begin{listing}[h]
\caption{The Dirac Delta function}\label{code:dirac}
\begin{minted}{python}
def DiracDelta(x: float) -> float:
	if x == 0:
		return float("infinity");
	return 0.
\end{minted}
\end{listing}

The expressions $x=0$, $x\ne0$, and equivalently \code{x == 0} are all actually simple functions that have an input of some real number and output boolean values.

\subsection{Boolean Operators}
Just like $=$, all of the other operators described in \ref{sec:relational-operators} are "infix" (meaning "goes in between operands") operators that take two inputs and produce a boolean output.
This is a comprehensive list of boolean operators not already described in \ref{sec:relational-operators} and their definitions, as well as examples.
